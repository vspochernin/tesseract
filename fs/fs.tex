\documentclass[a4paper, 14pt]{article}
\usepackage[margin=1.6cm]{geometry}
\usepackage[utf8]{inputenc}
\usepackage{minted}
\usepackage[russian]{babel}
\usepackage{amsmath}
\usepackage{graphicx}
\usepackage{changepage}
\usepackage{hyperref}
\usepackage{cases}
\usepackage{tikz-timing}[2017/12/20]
\usepackage{relsize}
\usepackage{booktabs}
\usepackage{gensymb}
\usepackage{multirow}
\usepackage{longtable}
\usetikzlibrary {arrows.meta}

\hypersetup{
	linkbordercolor = {1 1 1}
}

\usepackage{tikz-timing}[2009/05/15]
\usepackage{multicol}
\usepackage[T2A]{fontenc}
\usepackage{pgfplots}
%\usepackage[left=2.5cm, right=1.5cm, vmargin=2.5cm]{geometry}
\setlength\parindent{0pt} % Удалить отступы из параграфов.

\usepackage{listings}
\usepackage{caption}
\DeclareCaptionFont{white}{\color{white}} % Текст заголовка.
\DeclareCaptionFormat{listing}{\colorbox{gray}{\parbox{\textwidth}{#1#2#3}}}
\captionsetup[lstlisting]{format=listing,labelfont=white,textfont=white}
\renewcommand\labelenumi{\theenumi)}
\setlength\parindent{24pt}



\begin{document}
\lstset{
    language=java,                 % Выбор языка для подсветки (здесь это java).
    basicstyle=\small\sffamily,    % Размер и начертание шрифта для подсветки кода.
    numbers=left,                  % Где поставить нумерацию строк (слева\справа).
    numberstyle=\tiny,             % Размер шрифта для номеров строк.
    stepnumber=1,                  % Размер шага между двумя номерами строк.
    firstnumber=1,
    numberfirstline=true
    numbersep=5pt,                 % Как далеко отстоят номера строк от подсвечиваемого кода.
    backgroundcolor=\color{white}, % Цвет фона подсветки - используем \usepackage{color}.
    showspaces=false,              % Показывать или нет пробелы специальными отступами.
    showstringspaces=false,        % Показывать или нет пробелы в строках.
    showtabs=false,                % Показывать или нет табуляцию в строках.
    frame=single,                  % Рисовать рамку вокруг кода.
    tabsize=2,                     % Размер табуляции по умолчанию равен 2 пробелам.
    captionpos=t,                  % Позиция заголовка вверху [t] или внизу [b].
    breaklines=true,               % Автоматически переносить строки (да\нет).
    breakatwhitespace=false,       % Переносить строки только если есть пробел.
    escapeinside={\%*}{*)}         % Если нужно добавить комментарии в коде.
}

\begin{titlepage}
    \center

    ФЕДЕРАЛЬНОЕ ГОСУДАРСТВЕННОЕ АВТОНОМНОЕ ОБРАЗОВАТЕЛЬНОЕ УЧРЕЖДЕНИЕ ВЫСШЕГО ОБРАЗОВАНИЯ\linebreak
    «Санкт-Петербургский политехнический университет Петра Великого»
    \noindent\rule{500pt}{0.8pt} \\
    \textsc{\Large Институт компьютерных наук и кибербезопасности}\\
    \textsc{\large Высшая школа программной инженерии}\\[1.5cm]

    { \huge \bfseries КУРСОВОЙ ПРОЕКТ	\\
    \Large \mdseries АГРЕГАТОР ЦИФРОВЫХ ФИНАНСОВЫХ АКТИВОВ <<ТЕССЕРАКТ>> \\
    \large по дисциплине <<Технологии разработки качественного программного обеспечения>>}\\
    \flushright{
        {\phantom{qwe}}\\[1.0cm]
    }

    \begin{figure}[H]
        \centering
        \includegraphics[width=10cm]{1.png}\\[2.0cm]
    \end{figure}

    \begin{multicols}{2}
        \begin{flushright} \large

            {Выполнили студенты группы: 5130904/00104:}\\
            {\phantom{qwe}}\\
            {\phantom{qwe}}\\
            {\phantom{qwe}}\\
            {\phantom{qwe}}\\

            {Преподаватель:\\}

        \end{flushright}
        \begin{flushright}

            {Почернин В. С.}\\
            {Шиляев В. С.}\\
            {Мурзаканов И. М.}\\
            {Разукрантов А. А.}\\[0.5cm]


            Маслаков А. П.\\

        \end{flushright}
    \end{multicols}

    \flushright{
        {\phantom{qwe}}\\[0.5cm]
    }
    \centering{
        Санкт-Петербург\\
        2023
    }

    \vfill
\end{titlepage}

\Large
\tableofcontents
\newpage
\large

\section{Функциональная спецификация}

\subsection{Список страниц}

\begin{enumerate}
    \item Страница входа (LoginPage).
    \item Страница информации (InfoPage).
    \item Страница регистрации (RegistrationPage).
    \item Страница активов (AssetsPage).
    \item Страница конкретного актива (AssetPage).
    \item Страница избранных активов (FavouritesPage).
    \item Страница диверсификаций (DiversificationsPage).
    \item Страница создания диверсификации (DiversificationCreatePage).
    \item Страница конкретной диверсификации (DiversificationPage).
    \item Страница настроек (SettingsPage).
\end{enumerate}

\subsection{Функциональные требования}

\begin{longtable}{| p{0.3\textwidth} | p{0.7\textwidth} |}
    \hline
    \textbf{Идентификатор}       & \textbf{Требование}                                                                                                                                                 \\
    \hline
    \endfirsthead
    \hline
    \textbf{Идентификатор}       & \textbf{Требование}                                                                                                                                                 \\
    \hline
    \endhead

    % You can also use the below code for more things
    % \hline
    % \endfoot
    % \hline
    % \endlastfoot

    F\_LoginPage                 & Страница входа содержит логотип приложения, кнопки Войти, Зарегистрироваться, Информация, поля Логин, Пароль.                                                       \\ \hline
    F\_LoginPage                 & При нажатии на кнопку Информация происходит переход на станицу информации.                                                                                          \\ \hline
    F\_LoginPage                 & Кнопка Войти неактивна, пока хотя бы одно из полей Логин или Пароль пустое.                                                                                         \\ \hline
    F\_LoginPage                 & При нажатии на кнопку Войти и возникновении ошибки аутентификации, на экране отображается соответствущее уведомление об ошибке.                                     \\ \hline
    F\_LoginPage                 & При нажатии на кнопку Войти и корректной аутентификации происходит переход на страницу активов.                                                                     \\ \hline
    F\_LoginPage                 & При нажатии на кнопку Регистрация происходит переход на страницу регистрации.                                                                                       \\ \hline

    F\_InfoPage                  & Страница информации содержит информацию о приложении, кнопку Назад.                                                                                                 \\ \hline
    F\_InfoPage                  & При нажатии на кнопку Назад будет открыт предыдущий экран.                                                                                                          \\ \hline

    F\_RegistrationPage          & Страница регистрации содержит кнопки Назад, Зарегистрироваться, поля Логин, Email, Пароль, Подтверждение пароля, необязательные поля данных.                        \\ \hline % TODO: что за поля данных?
    F\_RegistrationPage          & При нажатии на кнопку Назад будет открыт предыдущий экран.                                                                                                          \\ \hline
    F\_RegistrationPage          & Кнопка Зарегистрироваться неактивна, пока хотя бы одно из полей Логин, Email, Пароль, Подтверждение пароля пустое.                                                  \\ \hline
    F\_RegistrationPage          & Длина пароля должна быть шесть или больше символов.                                                                                                                 \\ \hline
    F\_RegistrationPage          & При нажатии на кнопку зарегистрироваться и возникновении ошибки регистрации, на экране отображается соответствующее уведомление об ошибке.                          \\ \hline
    F\_RegistrationPage          & При нажатии на кнопку зарегистрироваться и успешной регистрации происходит переход на страницу входа, а на экране отображается уведомление об успешной регистрации. \\ \hline

    F\_Navigation                & Между страницами активов, избранных активов, диверсификаций, настроек навигация происходит с помощью нижнего меню.                                                  \\ \hline

    F\_AssetsPage                & Страница активов содержит список всех цифровых активов.                                                                                                             \\ \hline
    F\_AssetsPage                & Каждый элемент списка (актив) содержит базовую информацию об активе (название, происхождение, стоимость, изменение стоимости).                                      \\ \hline
    F\_AssetsPage                & На каждом элементе списка (активе) есть кнопка В избранное, позволяющая добавить/удалить актив в избранное.                                                         \\ \hline
    F\_AssetsPage                & При нажатии на актив открывается страница конкретного актива.                                                                                                       \\ \hline

    F\_AssetPage                 & Станица актива содержит в себе кнопки Назад, В избранное, а также полную информацию об активе.                                                                      \\ \hline
    F\_AssetPage                 & При нажатии на кнопку Назад будет открыт предыдущий экран.                                                                                                          \\ \hline
    F\_AssetPage                 & При нажатии кнопки В избранное происходит добавление/удаление актива в избранное.                                                                                   \\ \hline

    F\_FavouritesPage            & Страница избранных активов полностью повторяет страницу активов, за исключением того, что там хранятся только избранные активы пользователя.                        \\ \hline

    F\_DiversificationPage       & Страница диверсификаций содержит в себе список диверсификаций пользователя.                                                                                         \\ \hline
    F\_DiversificationPage       & Каждый элемент списка (диверсификация), кроме первого содержит дату и время создания, сумму и количество активов диверсификации.                                    \\ \hline
    F\_DiversificationPage       & Первый элемент списка является кнопкой Создать диверсификацию, при нажатии на которую происходит переход на экран создания диверсификации.                          \\ \hline
    F\_DiversificationPage       & При нажатии на не первый элемент списка (диверсификацию) происходит переход на страницу конкретной диверсификации.                                                  \\ \hline

    F\_DiversificationCreatePage & Страница создания диверсификации содержит в себе кнопки Назад, Создать диверсификацию, поля Сумма, Количество активов, выбор рискованности.                         \\ \hline
    F\_DiversificationCreatePage & При нажатии на кнопку Назад будет открыт предыдущий экран.                                                                                                          \\ \hline
    F\_DiversificationCreatePage & Кнопка Создать диверсификацию неактивна, пока хотя бы одно из полей Сумма, Количество активов пустое.                                                               \\ \hline
    F\_DiversificationCreatePage & Выбор рискованности является радио-кнопкой, состоящей из выборов Высокорискованная, Среднерискованная, Консервативная, Комбинированная.                             \\ \hline
    F\_DiversificationCreatePage & При нажатии на кнопку Создать диверсификацию и возникновении ошибки (некорректные данные), на экране отображается соответствующее уведомление об ошибке.            \\ \hline
    F\_DiversificationCreatePage & При нажатии на кнопку Создать диверсификацию и корректном создании, происходит переход на страницу созданной диверсификации.                                        \\ \hline
    \caption{Функциональные требования}
\end{longtable}

\subsection{Нефункциональные требования}

\end{document}