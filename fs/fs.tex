\documentclass[a4paper, 14pt]{article}
\usepackage[margin=1.6cm]{geometry}
\usepackage[utf8]{inputenc}
\usepackage{minted}
\usepackage[russian]{babel}
\usepackage{amsmath}
\usepackage{graphicx}
\usepackage{changepage}
\usepackage{hyperref}
\usepackage{cases}
\usepackage{tikz-timing}[2017/12/20]
\usepackage{relsize}
\usepackage{booktabs}
\usepackage{gensymb}
\usepackage{multirow}
\usepackage{longtable}
\usetikzlibrary {arrows.meta}

\hypersetup{
	linkbordercolor = {1 1 1}
}

\usepackage{tikz-timing}[2009/05/15]
\usepackage{multicol}
\usepackage[T2A]{fontenc}
\usepackage{pgfplots}
%\usepackage[left=2.5cm, right=1.5cm, vmargin=2.5cm]{geometry}
\setlength\parindent{0pt} % Удалить отступы из параграфов.

\usepackage{listings}
\usepackage{caption}
\DeclareCaptionFont{white}{\color{white}} % Текст заголовка.
\DeclareCaptionFormat{listing}{\colorbox{gray}{\parbox{\textwidth}{#1#2#3}}}
\captionsetup[lstlisting]{format=listing,labelfont=white,textfont=white}
\renewcommand\labelenumi{\theenumi)}
\setlength\parindent{24pt}



\begin{document}
\lstset{
    language=java,                 % Выбор языка для подсветки (здесь это java).
    basicstyle=\small\sffamily,    % Размер и начертание шрифта для подсветки кода.
    numbers=left,                  % Где поставить нумерацию строк (слева\справа).
    numberstyle=\tiny,             % Размер шрифта для номеров строк.
    stepnumber=1,                  % Размер шага между двумя номерами строк.
    firstnumber=1,
    numberfirstline=true
    numbersep=5pt,                 % Как далеко отстоят номера строк от подсвечиваемого кода.
    backgroundcolor=\color{white}, % Цвет фона подсветки - используем \usepackage{color}.
    showspaces=false,              % Показывать или нет пробелы специальными отступами.
    showstringspaces=false,        % Показывать или нет пробелы в строках.
    showtabs=false,                % Показывать или нет табуляцию в строках.
    frame=single,                  % Рисовать рамку вокруг кода.
    tabsize=2,                     % Размер табуляции по умолчанию равен 2 пробелам.
    captionpos=t,                  % Позиция заголовка вверху [t] или внизу [b].
    breaklines=true,               % Автоматически переносить строки (да\нет).
    breakatwhitespace=false,       % Переносить строки только если есть пробел.
    escapeinside={\%*}{*)}         % Если нужно добавить комментарии в коде.
}

\begin{titlepage}
    \center

    ФЕДЕРАЛЬНОЕ ГОСУДАРСТВЕННОЕ АВТОНОМНОЕ ОБРАЗОВАТЕЛЬНОЕ УЧРЕЖДЕНИЕ ВЫСШЕГО ОБРАЗОВАНИЯ\linebreak
    «Санкт-Петербургский политехнический университет Петра Великого»
    \noindent\rule{500pt}{0.8pt} \\
    \textsc{\Large Институт компьютерных наук и кибербезопасности}\\
    \textsc{\large Высшая школа программной инженерии}\\[1.5cm]

    { \huge \bfseries ФУНКЦИОНАЛЬНАЯ СПЕЦИФИКАЦИЯ	\\
    \Large \mdseries АГРЕГАТОР ЦИФРОВЫХ ФИНАНСОВЫХ АКТИВОВ <<ТЕССЕРАКТ>> \\
    \large по дисциплине <<Технологии разработки качественного программного обеспечения>>}\\
    \flushright{
        {\phantom{qwe}}\\[1.0cm]
    }

    \begin{figure}[H]
        \centering
        \includegraphics[width=10cm]{1.png}\\[2.0cm]
    \end{figure}

    \begin{multicols}{2}
        \begin{flushright} \large

            {Выполнили студенты группы: 5130904/00104:}\\
            {\phantom{qwe}}\\
            {\phantom{qwe}}\\
            {\phantom{qwe}}\\
            {\phantom{qwe}}\\

            {Преподаватель:\\}

        \end{flushright}
        \begin{flushright}

            {Почернин В. С.}\\
            {Шиляев В. С.}\\
            {Мурзаканов И. М.}\\
            {Разукрантов В. Е.}\\[0.5cm]


            Маслаков А. П.\\

        \end{flushright}
    \end{multicols}

    \flushright{
        {\phantom{qwe}}\\[0.5cm]
    }
    \centering{
        Санкт-Петербург\\
        2023
    }

    \vfill
\end{titlepage}

\Large
\tableofcontents
\newpage
\large

\section{Функциональная спецификация}

\subsection{Список страниц}

\begin{enumerate}
    \item Страница входа (LoginPage).
    \item Страница информации (InfoPage).
    \item Страница регистрации (RegistrationPage).
    \item Страница активов (AssetsPage).
    \item Страница конкретного актива (AssetPage).
    \item Страница избранных активов (FavouritesPage).
    \item Страница диверсификаций (DiversificationsPage).
    \item Страница создания диверсификации (DiversificationCreatePage).
    \item Страница конкретной диверсификации (DiversificationPage).
    \item Страница настроек (SettingsPage).
\end{enumerate}

\subsection{Функциональные требования}

\begin{longtable}{| p{0.35\textwidth} | p{0.65\textwidth} |}
    \hline
    \textbf{Идентификатор}          & \textbf{Требование}                                                                                                                                                                \\
    \hline
    \endfirsthead
    \hline
    \textbf{Идентификатор}          & \textbf{Требование}                                                                                                                                                                \\
    \hline
    \endhead

    % You can also use the below code for more things
    % \hline
    % \endfoot
    % \hline
    % \endlastfoot

    F\_LoginPage\_1                 & Страница входа содержит логотип приложения, кнопки Войти, Войти через Google, Зарегистрироваться, Информация, поля Логин, Пароль.                                                  \\ \hline
    F\_LoginPage\_2                 & При нажатии на кнопку Информация происходит переход на станицу информации.                                                                                                         \\ \hline
    F\_LoginPage\_3                 & Кнопка Войти неактивна, пока хотя бы одно из полей Логин или Пароль пустое.                                                                                                        \\ \hline
    F\_LoginPage\_4                 & При нажатии на кнопку Войти и возникновении ошибки аутентификации, на экране отображается соответствущее уведомление об ошибке.                                                    \\ \hline
    F\_LoginPage\_5                 & При нажатии на кнопку Войти и корректной аутентификации происходит переход на страницу активов.                                                                                    \\ \hline
    F\_LoginPage\_6                 & При нажатии на кнопку Войти через Google и возникновении ошибки аутентификации, на экране отображается соответствущее уведомление об ошибке.                                       \\ \hline
    F\_LoginPage\_7                 & При нажатии на кнопку Войти через Google и корректной аутентификации через Google OAuth API происходит переход на страницу активов.                                                \\ \hline
    F\_LoginPage\_8                 & При нажатии на кнопку Регистрация происходит переход на страницу регистрации.                                                                                                      \\ \hline

    F\_InfoPage\_1                  & Страница информации содержит информацию о приложении, кнопку Назад.                                                                                                                \\ \hline
    F\_InfoPage\_2                  & При нажатии на кнопку Назад будет открыт предыдущий экран.                                                                                                                         \\ \hline

    F\_RegistrationPage\_1          & Страница регистрации содержит кнопки Назад, Зарегистрироваться, поля Логин, Email, Пароль, Подтверждение пароля.                                                                   \\ \hline
    F\_RegistrationPage\_2          & При нажатии на кнопку Назад будет открыт предыдущий экран.                                                                                                                         \\ \hline
    F\_RegistrationPage\_3          & Кнопка Зарегистрироваться неактивна, пока хотя бы одно из полей Логин, Email, Пароль, Подтверждение пароля пустое.                                                                 \\ \hline
    F\_RegistrationPage\_4          & Длина пароля должна составлять от 6ти до 30ти символов включительно.                                                                                                               \\ \hline
    F\_RegistrationPage\_5          & Пароль может состоять из цифр, латинских букв верхнего и нижнего регистра, а также специальных символов.                                                                           \\ \hline
    F\_RegistrationPage\_6          & Специальными символами являются: <<! @ \# \$ \% \& * ( ) - \_ + = ; : , . / ? $\backslash$ | [ ] \{ \}>>.                                                                          \\ \hline
    F\_RegistrationPage\_7          & Пароль должен содержать хотя бы одну букву (в любом регистре), а также хотя бы одну цифру или специальны символ (т.е. Буква И (Цифра ИЛИ Специальный символ)).                     \\ \hline
    F\_RegistrationPage\_8          & При нажатии на кнопку зарегистрироваться и возникновении ошибки регистрации, на экране отображается соответствующее уведомление об ошибке.                                         \\ \hline
    F\_RegistrationPage\_9          & При нажатии на кнопку зарегистрироваться и успешной регистрации происходит переход на страницу входа, а на экране отображается уведомление об успешной регистрации.                \\ \hline
    F\_RegistrationPage\_10         & Длина логина должна составлять от 3х до 16ти символов включительно.                                                                                                                \\ \hline
    F\_RegistrationPage\_11         & Логин может состоять из цифр и латинских букв верхнего и нижнего регистра.                                                                                                         \\ \hline

    F\_Navigation\_1                & Между страницами активов, избранных активов, диверсификаций, настроек навигация происходит с помощью нижнего меню.                                                                 \\ \hline

    F\_AssetsPage\_1                & Страница активов содержит список всех цифровых активов.                                                                                                                            \\ \hline
    F\_AssetsPage\_2                & Каждый элемент списка (актив) содержит базовую информацию об активе (название, происхождение, стоимость, изменение стоимости).                                                     \\ \hline
    F\_AssetsPage\_3                & На каждом элементе списка (активе) есть кнопка В избранное, позволяющая добавить/удалить актив в избранное.                                                                        \\ \hline
    F\_AssetsPage\_4                & При нажатии на актив открывается страница конкретного актива.                                                                                                                      \\ \hline
    F\_AssetsPage\_5                & Страница активов единовременно отображает на экране не более 5ти полных элементов списка с возможностями прокручивания и динамической загрузки.                                    \\ \hline

    F\_AssetPage\_1                 & Страница актива содержит в себе кнопки Назад, В избранное, а также полную информацию об активе.                                                                                    \\ \hline
    F\_AssetPage\_2                 & При нажатии на кнопку Назад будет открыт предыдущий экран.                                                                                                                         \\ \hline
    F\_AssetPage\_3                 & При нажатии кнопки В избранное происходит добавление/удаление актива в избранное.                                                                                                  \\ \hline

    F\_FavouritesPage\_1            & Страница избранных активов полностью повторяет страницу активов, за исключением того, что там хранятся только избранные активы пользователя.                                       \\ \hline

    F\_DiversificationsPage\_1      & Страница диверсификаций содержит в себе список диверсификаций пользователя.                                                                                                        \\ \hline
    F\_DiversificationsPage\_2      & Каждый элемент списка (диверсификация), кроме первого содержит дату и время создания, сумму и количество активов диверсификации.                                                   \\ \hline
    F\_DiversificationsPage\_3      & Первый элемент списка является кнопкой Создать диверсификацию, при нажатии на которую происходит переход на экран создания диверсификации.                                         \\ \hline
    F\_DiversificationsPage\_4      & При нажатии на не первый элемент списка (диверсификацию) происходит переход на страницу конкретной диверсификации.                                                                 \\ \hline

    F\_DiversificationCreatePage\_1 & Страница создания диверсификации содержит в себе кнопки Назад, Создать диверсификацию, поля Сумма, Количество активов, Степень рискованности.                                      \\ \hline
    F\_DiversificationCreatePage\_2 & При нажатии на кнопку Назад будет открыт предыдущий экран.                                                                                                                         \\ \hline
    F\_DiversificationCreatePage\_3 & Кнопка Создать диверсификацию неактивна, пока хотя бы одно из полей Сумма, Количество активов пустое.                                                                              \\ \hline
    F\_DiversificationCreatePage\_4 & Выбор рискованности является радио-кнопкой, состоящей из выборов Высокорискованная, Среднерискованная, Консервативная, Комбинированная.                                            \\ \hline
    F\_DiversificationCreatePage\_5 & При нажатии на кнопку Создать диверсификацию и возникновении ошибки (некорректные данные), на экране отображается соответствующее уведомление об ошибке.                           \\ \hline
    F\_DiversificationCreatePage\_6 & При нажатии на кнопку Создать диверсификацию и корректном создании, происходит переход на страницу созданной диверсификации.                                                       \\ \hline
    F\_DiversificationCreatePage\_7 & Страница диверсификаций единовременно отображает на экране не более 5ти полных элементов списка с возможностями прокручивания и динамической загрузки.                              \\ \hline

    F\_DiversificationPage\_1       & Страница диверсификации содержит в себе кнопку Назад, а также полную информацию о диверсификации (Сумму, Количество активов, Степень рискованности, список всех цифровых активов). \\ \hline
    F\_DiversificationPage\_2       & Страница диверсификации содержит в себе кнопку Назад, а также полную информацию о диверсификации (Сумму, Количество активов, Степень рискованности, список всех цифровых активов). \\ \hline
    F\_DiversificationPage\_3       & При нажатии на кнопку Назад будет открыт предыдущий экран.                                                                                                                         \\ \hline
    F\_DiversificationPage\_4       & При нажатии на какой-либо актив будет совершен переход на страницу конкретного актива.                                                                                             \\ \hline

    F\_SettingsPage\_1              & Экран настроек содержит в себе кнопки Назад, Изменить пароль, Переключить тему, поля Cтарый пароль, Новый пароль, Подтверждение нового пароля.                                     \\ \hline
    F\_SettingsPage\_2              & При нажатии на кнопку Назад будет открыт предыдущий экран.                                                                                                                         \\ \hline
    F\_SettingsPage\_3              & Кнопка изменить пароль неактивна, если хотя бы одно из полей Старый пароль, Новый пароль, Подтверждение нового пароля пустое.                                                      \\ \hline
    F\_SettingsPage\_4              & При нажатии на кнопку Изменить пароль и ошибки изменения пароля, на экране отображается соответствующее уведомление об ошибке.                                                     \\ \hline
    F\_SettingsPage\_5              & При нажатии на кнопку Изменить пароль и успешном изменении пароля, на экране отображается соответствующее уведомление.                                                             \\ \hline
    F\_SettingsPage\_6              & При нажатии на кнопку Переключить тему - меняется тема приложения (светлая, темная).                                                                                               \\ \hline
    \caption{Функциональные требования}
\end{longtable}

\subsection{Нефункциональные требования}

\begin{longtable}{| p{0.3\textwidth} | p{0.7\textwidth} |}
    \hline
    \textbf{Идентификатор} & \textbf{Требование}                                                                                               \\
    \hline
    \endfirsthead
    \hline
    \textbf{Идентификатор} & \textbf{Требование}                                                                                               \\
    \hline
    \endhead

    % You can also use the below code for more things
    % \hline
    % \endfoot
    % \hline
    % \endlastfoot

    NF\_PasswordStore      & Пароли должны храниться в базе данных в зашифрованном виде.                                                       \\ \hline
    NF\_ClientServer       & Приложение должно иметь клиент-серверную архитектуру.                                                             \\ \hline
    NF\_OS                 & Приложение должно работать на мобильной ОС Android.                                                               \\ \hline
    NF\_AssetsStore        & Хранение информации об акивах должно иметь унифицированный вид для возможности удобного добавления новых активов. \\ \hline
    NF\_Design             & Внешний вид мобильного приложения должен быь выполнен в соответствии с Material Design 3.                         \\ \hline
    NF\_Language           & Приложение должно работать либо на русском, либо на английском языке, в зависимости от настроек системы.          \\ \hline
    NF\_ServerLanguage     & Серверная часть приложения должна быть написана на языке Java.                                                    \\ \hline
    \caption{Нефункциональные требования}
\end{longtable}

\end{document}