\documentclass[a4paper, 14pt]{article}
\usepackage[margin=1.6cm]{geometry}
\usepackage[utf8]{inputenc}
\usepackage[russian]{babel}
\usepackage{amsmath}
\usepackage{graphicx}
\usepackage{changepage}
\usepackage{hyperref}
\usepackage{cases}
\usepackage{tikz-timing}[2017/12/20]
\usepackage{relsize}
\usepackage{booktabs}
\usepackage{gensymb}
\usepackage{multirow}
\usepackage{longtable}
\usetikzlibrary {arrows.meta}

\hypersetup{
	linkbordercolor = {1 1 1}
}

\usepackage{tikz-timing}[2009/05/15]
\usepackage{multicol}
\usepackage[T2A]{fontenc}
\usepackage{pgfplots}
%\usepackage[left=2.5cm, right=1.5cm, vmargin=2.5cm]{geometry}
\setlength\parindent{0pt} % Удалить отступы из параграфов.

\usepackage{listings}
\usepackage{caption}
\DeclareCaptionFont{white}{\color{white}} % Текст заголовка.
\DeclareCaptionFormat{listing}{\colorbox{gray}{\parbox{\textwidth}{#1#2#3}}}
\captionsetup[lstlisting]{format=listing,labelfont=white,textfont=white}
\renewcommand\labelenumi{\theenumi)}
\setlength\parindent{24pt}



\begin{document}
\lstset{
    language=java,                 % Выбор языка для подсветки (здесь это java).
    basicstyle=\small\sffamily,    % Размер и начертание шрифта для подсветки кода.
    numbers=left,                  % Где поставить нумерацию строк (слева\справа).
    numberstyle=\tiny,             % Размер шрифта для номеров строк.
    stepnumber=1,                  % Размер шага между двумя номерами строк.
    firstnumber=1,
    numberfirstline=true
    numbersep=5pt,                 % Как далеко отстоят номера строк от подсвечиваемого кода.
    backgroundcolor=\color{white}, % Цвет фона подсветки - используем \usepackage{color}.
    showspaces=false,              % Показывать или нет пробелы специальными отступами.
    showstringspaces=false,        % Показывать или нет пробелы в строках.
    showtabs=false,                % Показывать или нет табуляцию в строках.
    frame=single,                  % Рисовать рамку вокруг кода.
    tabsize=2,                     % Размер табуляции по умолчанию равен 2 пробелам.
    captionpos=t,                  % Позиция заголовка вверху [t] или внизу [b].
    breaklines=true,               % Автоматически переносить строки (да\нет).
    breakatwhitespace=false,       % Переносить строки только если есть пробел.
    escapeinside={\%*}{*)}         % Если нужно добавить комментарии в коде.
}

\begin{titlepage}
    \center

    ФЕДЕРАЛЬНОЕ ГОСУДАРСТВЕННОЕ АВТОНОМНОЕ ОБРАЗОВАТЕЛЬНОЕ УЧРЕЖДЕНИЕ ВЫСШЕГО ОБРАЗОВАНИЯ\linebreak
    «Санкт-Петербургский политехнический университет Петра Великого»
    \noindent\rule{500pt}{0.8pt} \\
    \textsc{\Large Институт компьютерных наук и кибербезопасности}\\
    \textsc{\large Высшая школа программной инженерии}\\[8.0cm]

    { \huge \bfseries ПРАКТИЧЕСКАЯ РАБОТА №2	\\
    \Large \mdseries <<ГОСТ 19.201-78>> \\
    \large по дисциплине <<Авторское право, метрология и стандартизация программного обеспечения>>}\\
    \flushright{
        {\phantom{qwe}}\\[8.0cm]
    }

    \begin{multicols}{2}
        \begin{flushright} \large

            {Выполнили студенты группы: 5130904/00104:}\\
            {\phantom{qwe}}\\
            {\phantom{qwe}}\\
            {\phantom{qwe}}\\
            {\phantom{qwe}}\\

            {Преподаватель:\\}

        \end{flushright}
        \begin{flushright}

            {Почернин В. С.}\\
            {Шиляев В. С.}\\
            {Мурзаканов И. М.}\\
            {Разукрантов В. Е.}\\[0.5cm]


            Юсупова О. А.\\

        \end{flushright}
    \end{multicols}

    \flushright{
        {\phantom{qwe}}\\[0.5cm]
    }
    \centering{
        Санкт-Петербург\\
        2024
    }

    \vfill
\end{titlepage}

\Large
\tableofcontents
\large

\newpage
\section{Постановка задачи}

Необходимо:

\begin{itemize}
    \item Составить техническое задание выбранного проекта по \texttt{ГОСТ 19.201-78}.
    \item Заполнить все разделы.
    \item Оформить титульный лист по правилам \texttt{СПбГПУ}, текст отчета оформить в соответствии с \texttt{ГОСТ 19.201-78}.
\end{itemize}

\newpage
\section{Теория}

\subsection{Международные стандарты}

\begin{itemize}
    \item \texttt{IEEE Std 1063-2001} <<IEEE Standard for Software User Documentation>> - стандарт для написания руководства пользователя.
    \item \texttt{ISO/IEC 26514:2008} <<Requirements for designers and developers of user documentation>> - стандарт для дизайнеров и разработчиков пользовательской документации.
    \item \texttt{IEEE Std 1016-1998} <<Recommended Practice for Software Design Descriptions>> - стандарт для написания технического описания программы.
    \item \texttt{ISO/IEC FDIS 18019:2004} <<Guidelines for the design and preparation of user documentation for application software>> - стандарт для написания руководства пользователя.
\end{itemize}

\subsection{Российские стандарты}

\begin{itemize}
    \item \texttt{ГОСТ} (государственный стандарт) - это нормативно-правовой документ, в соответствии с требованиями которого производится стандартизация производственных процессов и оказания услуг.
    \item Государственный стандарт обязательно проходит процедуру регистрации, которая проводится специальным государственным органом - \texttt{Госстандартом}.
    \item \texttt{ГОСТы} могут заменяться или отменяться. Действующие \texttt{ГОСТы} обязательны к исполнению.
    \item Утвержденный \texttt{ГОСТ} содержит ключевые требования, которым должны соответствовать товары, работы и услуги, в отношении которых он принимается, для обеспечения их эффективной и безопасной эксплуатации.
    \item \textbf{\texttt{ГОСТ 19} Единая система программной документации (\texttt{ЕСПД})}.
    \item \textbf{\texttt{ГОСТ 34} Информационные технологии. Комплекс стандартов на автоматизированные системы}.
\end{itemize}

\subsection{ГОСТ 34}

\begin{itemize}
    \item Автоматизированная система (\texttt{АС}) - это система, состоящая из персонала и комплекса средств автоматизации его деятельности, реализующая информационную технологию выполнения установленных функций. В зависимости от вида деятельности выделяют следующие виды \texttt{АС}:
    \begin{itemize}
        \item Автоматизированные системы управления (\texttt{АСУ}).
        \item Системы автоматизированного проектирования (\texttt{САПР}).
        \item Автоматизированные системы научных исследований (\texttt{АСНИ}).
        \item Другие.
    \end{itemize}
    \item \texttt{ГОСТ 34} разделяет виды обеспечения АС:
    \begin{itemize}
        \item Организационное.
        \item Методическое.
        \item Техническое.
        \item Математическое.
        \item Программное обеспечение.
        \item Информационное.
        \item Лингвистическое.
        \item Правовое.
        \item Эргономическое.
    \end{itemize}
    \item Автоматизированная система - это не программа, а целый комплекс видос обеспечения.
    \item \texttt{34.201-89} Виды, комплектность и обозначения документов при создании автоматизированных систем.
    \item \texttt{34.320-96} Концепции и терминология для концептуальной схемы и информационной базы.
    \item \texttt{34.321-96} Информационные технологии. Система стандартов по базам данных. Эталонная модель управления.
    \item \texttt{34.601-90} Автоматизированные системы. Стадии создания.
    \item \texttt{34.602-89} Техническое задание на создание автоматизированной системы.
    \item \texttt{34.603-92} Информационная технология. Виды испытаний автоматизированных систем.
    \item \texttt{РД 50-34.698-90} Автоматизированные системы. Требования к содержанию документов.
    \item \texttt{Р ИСО/МЭК 8824-3-2002} Информационная технология. Абстрактная синтаксическая нотация.
    \item \texttt{Р ИСО/МЭК 10746-3-2001} Управление данными и открытая распределенная обработка.
\end{itemize}

\subsection{ГОСТ 19}

\begin{itemize}
    \item \texttt{ЕСПД} (единая система программной документации) - комплекс государственных стандартов (ГОСТ), устанавливающих взаимосвязанные правила разработки, оформления и обращения программ и программной документации.
    \item Стандарты ЕСПД устанавливают требования, регламентирующие разработку, сопровождение, изготовление и эксплуатацию программ.
    \item Определения из \texttt{ЕСПД}:
    \begin{itemize}
        \item Программа - данные, предназначенные для управления конкретными компонентами системы обработки информации в целях реализации определенного алгоритма.
        \item Программное обеспечение - совокупность программ системы обработки информации и программных документов, необходимых для эксплуатации этих программ.
    \end{itemize}
    \item \texttt{19.001-77} Общие положения.
    \item \texttt{19.005-85} Р-схемы алгоритмов и программ. Обозначения условные и графические и правила выполнения.
    \item \texttt{19.101-77} Виды программ и программных документов.
    \item \texttt{19.102-77} Стадии разработки.
    \item \texttt{19.103-77} Обозначения программ и программных документов.
    \item \texttt{19.104-78} Основные надписи.
    \item \texttt{19.105-78} Общие требования к программным документам.
    \item \texttt{19.106-78} Требования к программным документам, выполненным печатным способом.
    \item \textbf{\texttt{19.201-78} Техническое задание. Требования к содержанию и оформлению}.
    \item \texttt{19.202-78} Спецификация. Требования к содержанию и оформлению.
    \item \texttt{19.301-79} Программа и методика испытаний. Требования к содержанию и оформлению.
    \item \texttt{19.401-78} Текст программы. Требования к содержанию и оформлению.
    \item \texttt{19.402-78} Описание программы.
    \item \texttt{19.403-79} Ведомость держателей подлинников.
    \item \texttt{19.404-79} Пояснительная записка. Требования к содержанию и оформлению.
    \item \texttt{19.501-78} Формуляр. Требования к содержанию и оформлению.
    \item \texttt{19.502-78} Описание применения. Требования к содержанию и оформлению.
    \item \texttt{19.503-79} Руководство системного программиста. требования к содержанию и оформлению.
    \item \texttt{19.504-79} Руководство программиста. Требования к содержанию и оформлению.
    \item \texttt{19.505-79} Руководство оператора. Требования к содержанию и оформлению.
    \item \texttt{19.506-79} Описание языка. Требования к содержанию и оформлению.
    \item \texttt{19.507-79} Ведомость эксплуатационных документов.
    \item \texttt{19.508-79} Руководство по техническому обслуживанию. Требования к содержанию и оформлению.
    \item \texttt{19.601-78} Общие правила дублирования, учета и хранения.
    \item \texttt{19.602-78} Правила дублирования, учета и хранения программных документов, выполненных печатным способом.
    \item \texttt{19.603-78} Общие правила внесения изменений.
    \item \texttt{19.604-78} Правила внесения изменений в программные документы, выполненные печатным способом.
    \item \texttt{19.701-90 (ИСО 5807-85)} Схемы алгоритмов, программ, данных и систем. Обозначения условные и правила выполнения.
    \item \texttt{19.781-90} Обеспечение систем обработки информации программное. Термины и определения.
\end{itemize}

\subsection{ГОСТ 19.201-78 ЕСПД. Техническое задание. Требования к содержанию и оформлению}

Настоящий стандарт устанавливает порядок построения и оформления технического задания на разработку программы или программного изделия для вычислительных машин, комплексов и систем независимо от их назначения и области применения.

\subsubsection{Общие положения}

\begin{enumerate}
    \item Техническое задания оформляют в соответствии с \texttt{ГОСТ 19.106-78} на листах формата 11 и 12 по \texttt{ГОСТ 2.301-68}, как правило, без заполнения полей листа. Номера листов (страниц) проставляют в верхней части листа над текстом.
    \item Лист утверждения и титульный лист оформляют в соответствии с \texttt{ГОСТ 19.104-78}. Информационную часть (аннотацию и содержание), лист регистрации изменений допускается в документ не включать.
    \item Для внесения изменений или дополнений в техническое задание на последующих стадиях разработки программы или программного изделия выпускают дополнение к нему. Согласование и утверждение дополнения к техническому заданию проводят в том же порядке, который установлен для технического задания.
    \item Техническое задание должно содержать следующие разделы:
    \begin{itemize}
        \item Введение.
        \item Основания для разработки.
        \item Назначение разработки.
        \item Требования к программе или программному изделию.
        \item Требования к программной документации.
        \item Технико-экономические показатели.
        \item Стадии и этапы разработки.
        \item Порядок контроля и приемки.
    \end{itemize}
    В техническое задание допускается включать приложения. В зависимости от особенностей программы или программного изделия допускается уточнять содержания разделов, вводить новые разделы или объединять отдельные из них.
\end{enumerate}

\subsubsection{Содержания разделов}

\begin{enumerate}
    \item В разделе <<\texttt{Введение}>> указывают наименование, краткую характеристику области применения программы или программного изделия и объекта, в котором используют программу или программное изделие.
    \item В разделе <<\texttt{Основания для разработки}>> должны быть указаны:
    \begin{itemize}
        \item Документ (документы), на основании которых ведется разработка.
        \item Организация, утвердившая этот документ и дата его утверждения.
        \item Наименование и (или) условное обозначение темы разработки.
    \end{itemize}
    \item В разделе <<\texttt{Назначение разработки}>> должно быть указано функциональное и эксплуатационное назначение программы или программного изделия.
    \item Раздел <<\texttt{Требования к программе или программному изделию}>> должен содержать следующие подразделы:
    \begin{itemize}
        \item Требования к функциональным характеристикам.
        \item Требования к надежности.
        \item Условия эксплуатации.
        \item Требования к составу и параметрам технических средств.
        \item Требования к информационной и программной совместимости.
        \item Требования к маркировке и упаковке.
        \item Требования к транспортированию и хранению.
        \item Специальные требования.
    \end{itemize}
    \begin{enumerate}
        \item В подразделе <<\texttt{Требования к функциональным характеристикам}>> должны быть указаны требования к составу выполняемых функций, организации выходных и выходных данных, временным характеристикам и т. д.
        \item В подразделе <<\texttt{Требования к надежности}>> должны быть указаны требования к обеспечению надежного функционирования (обеспечение устойчивого функционирования, контроль входной и выходной информации, время восстановления после отказа и т. д.).
        \item В подразделе <<\texttt{Условия эксплуатации}>> должны быть указаны условия эксплуатации (температура окружающего воздуха, относительная влажность и т. д. для выбранных типов носителей данных), при которых должны обеспечиваться заданные характеристики, а также вид обслуживания, необходимое количество и квалификация персонала.
        \item В подразделе <<\texttt{Требования к составу и параметрам технических средств}>> указывают необходимый состав технических средств с указанием их основных технических характеристик.
        \item В подразделе <<\texttt{Требования к информационной и программной совместимости}>> должны быть указаны требования к информационным структурам на входе и выходе и методам решения, исходным кодам, языкам программирования и программным средствам, используемым программой. При необходимости должн обеспечиваться защита информации и программ.
        \item В подразделе <<\texttt{Требования к маркировке и упаковке}>> в общем случае указывают требования к маркировке программного изделия, варианты и способы упаковки.
        \item В подразделе <<\texttt{Требования к транспортированию и хранению}>> должны быть указаны для программного изделия условия транспортирования, места хранения, условия хранения, условия складирования, сроки хранения в различных условиях.
    \end{enumerate}
    \item В разделе <<\texttt{Требования к программной документации}>> должны быть указаны предварительный состав программной документации и, при необходимости, специальные требования к ней.
    \item В разделе <<\texttt{Технико-экономические показатели}>> должны быть указаны:
    \begin{itemize}
        \item Ориентировочная экономическая эффективность.
        \item Предполагаемая годовая потребность.
        \item Экономические преимущества разработки по сравнению с лучшими отечественными и зарубежными образцами или аналогами.
    \end{itemize}
    \item В разделе <<\texttt{Стадии и этапы разработки}>> устанавливают необходимые стадии разработки, этапы и содержание работа (перечень программных документов, которые должны быть разработаны, согласованы и утверждены), а также, как правило, сроки разработки и определяют исполнителей.
    \item В разделе <<\texttt{Порядок контроля и приемки}>> должны быть указаны виды испытаний и общие требования к приемке работы.
    \item В приложениях к техническому заданию, при необходимости приводят:
    \begin{itemize}
        \item Перечень научно-исследовательских и других работ, обосновывающих разработку.
        \item Схемы алгоритмов, таблицы, описания, обоснования, расчеты и другие документы, которые могут быть использованы при разработке.
        \item Другие источники разработки.
    \end{itemize}
\end{enumerate}

\newpage
\section{Ход работы}

\subsection{Введение}

\subsubsection{Наименование программы}

\begin{itemize}
    \item Наименование программы на английском языке: <<\texttt{Tesseract}>>.
    \item Наименование программы на русском языке: <<\texttt{Тессеракт}>>.
\end{itemize}

\subsubsection{Краткая характеристика области применения программы}

\begin{itemize}
    \item Приложение <<\texttt{Тессеракт}>> предназначено для агрегации цифровых финансовых активов от различных операторов информационных систем.
\end{itemize}

\subsection{Основания для разработки}

\begin{itemize}
    \item Основанием для разработки является Курсовая работа по дисциплине <<\texttt{Технологии разработки качественного программного обеспечения}>> под руководством старшего преподавателя \texttt{ВШПИ ИКНТ СПбПУ} Маслакова Алексея Павловича.
    \item Наименование темы разработки - <<\texttt{Агрегатор цифровых финансовых активов <<Тессеракт>>}>>.
\end{itemize}

\subsection{Назначение разработки}

\subsubsection{Функциональное назначение}

\begin{itemize}
    \item Функциональным назначением является осведомление пользователей о существовании цифровых финансовых активов от всех существующих операторов информационных систем, а также предоставление возможности автоматического создания портфелей инвестирования по заданным параметрам (стоимость портфеля, уровень рискованности активов в портфеле).
\end{itemize}

\subsubsection{Эксплуатационное назначение}

\begin{itemize}
    \item Эксплуатационным назначением является возможность запуска и использования приложения на мобильных устройствах, работающих на операционной системе \texttt{Android}.
\end{itemize}

\subsection{Требования к программе или программному изделию}

\subsubsection{Требования к функциональным характеристикам}

\begin{longtable}{| p{0.35\textwidth} | p{0.65\textwidth} |}
    \hline
    \textbf{Идентификатор}          & \textbf{Требование}                                                                                                                                                                \\
    \hline
    \endfirsthead
    \hline
    \textbf{Идентификатор}          & \textbf{Требование}                                                                                                                                                                \\
    \hline
    \endhead

    F\_LoginPage\_1                 & Страница входа содержит логотип приложения, кнопки Войти, Войти с помощью Google, Зарегистрироваться, Информация, поля Логин, Пароль. Dev-версия приложения допускает наличие кнопки Изменить API URL.                                                  \\ \hline
    F\_LoginPage\_2                 & При нажатии на кнопку Информация происходит переход на станицу информации.                                                                                                         \\ \hline
    F\_LoginPage\_3                 & Кнопка Войти неактивна, пока хотя бы одно из полей Логин или Пароль пустое.                                                                                                        \\ \hline
    F\_LoginPage\_4                 & При нажатии на кнопку Войти и возникновении ошибки аутентификации, на экране отображается соответствущее уведомление об ошибке.                                                    \\ \hline
    F\_LoginPage\_5                 & При нажатии на кнопку Войти и корректной аутентификации происходит переход на страницу активов.                                                                                    \\ \hline
    F\_LoginPage\_6                 & При нажатии на кнопку Войти через Google и возникновении ошибки аутентификации, на экране отображается соответствущее уведомление об ошибке.                                       \\ \hline
    F\_LoginPage\_7                 & При нажатии на кнопку Войти через Google и корректной аутентификации через Google OAuth API происходит переход на страницу активов.                                                \\ \hline
    F\_LoginPage\_8                 & При нажатии на кнопку Регистрация происходит переход на страницу регистрации.                                                                                                      \\ \hline
    F\_LoginPage\_9                 & При нажатии на кнопку Изменить API URL открывается окно, в котором можно установить URL сервера.                                                                                                      \\ \hline

    \caption{Функциональные требования для страницы входа}
\end{longtable}

\begin{longtable}{| p{0.35\textwidth} | p{0.65\textwidth} |}
    \hline
    \textbf{Идентификатор}          & \textbf{Требование}                                                                                                                                                                \\
    \hline
    \endfirsthead
    \hline
    \textbf{Идентификатор}          & \textbf{Требование}                                                                                                                                                                \\
    \hline
    \endhead

    F\_InfoPage\_1                  & Страница информации содержит информацию о приложении, кнопку Назад.                                                                                                                \\ \hline
    F\_InfoPage\_2                  & При нажатии на кнопку Назад будет открыт предыдущий экран.                                                                                                                         \\ \hline

    \caption{Функциональные требования для страницы информации}
\end{longtable}

\begin{longtable}{| p{0.35\textwidth} | p{0.65\textwidth} |}
    \hline
    \textbf{Идентификатор}          & \textbf{Требование}                                                                                                                                                                \\
    \hline
    \endfirsthead
    \hline
    \textbf{Идентификатор}          & \textbf{Требование}                                                                                                                                                                \\
    \hline
    \endhead

    F\_RegistrationPage\_1          & Страница регистрации содержит кнопки Назад, Зарегистрироваться, поля Логин, Email, Пароль, Подтверждение пароля.                                                                   \\ \hline
    F\_RegistrationPage\_2          & При нажатии на кнопку Назад будет открыт предыдущий экран.                                                                                                                         \\ \hline
    F\_RegistrationPage\_3          & Кнопка Зарегистрироваться неактивна, пока хотя бы одно из полей Логин, Email, Пароль, Подтверждение пароля пустое.                                                                 \\ \hline
    F\_RegistrationPage\_4          & Длина пароля должна составлять от 6ти до 30ти символов включительно.                                                                                                               \\ \hline
    F\_RegistrationPage\_5          & Пароль может состоять из цифр, латинских букв верхнего и нижнего регистра, а также специальных символов.                                                                           \\ \hline
    F\_RegistrationPage\_6          & Специальными символами являются: <<! @ \# \$ \% \& * ( ) - \_ + = ; : , . / ? $\backslash$ | [ ] \{ \}>>.                                                                          \\ \hline
    F\_RegistrationPage\_7          & Пароль должен содержать хотя бы одну букву (в любом регистре), а также хотя бы одну цифру или специальны символ (т.е. Буква И (Цифра ИЛИ Специальный символ)).                     \\ \hline
    F\_RegistrationPage\_8          & При нажатии на кнопку зарегистрироваться и возникновении ошибки регистрации, на экране отображается соответствующее уведомление об ошибке.                                         \\ \hline
    F\_RegistrationPage\_9          & При нажатии на кнопку зарегистрироваться и успешной регистрации происходит переход на страницу входа, а на экране отображается уведомление об успешной регистрации.                \\ \hline
    F\_RegistrationPage\_10         & Длина логина должна составлять от 3х до 16ти символов включительно.                                                                                                                \\ \hline
    F\_RegistrationPage\_11         & Логин может состоять из цифр и латинских букв верхнего и нижнего регистра.                                                                                                         \\ \hline
    F\_RegistrationPage\_12         & Логин должен быть уникальным.                                                                                                         \\ \hline
    F\_RegistrationPage\_13         & Email должен содержать имя пользователя, разделитель (@), имя почтового сервера.                                                                                                         \\ \hline
    F\_RegistrationPage\_14         & Email должен быть уникальным.                                                                                                         \\ \hline
    F\_RegistrationPage\_15         & Поля Пароль и Подтверждение пароля должны совпадать.                                                                                                         \\ \hline

    \caption{Функциональные требования для страницы регистрации}
\end{longtable}

\begin{longtable}{| p{0.35\textwidth} | p{0.65\textwidth} |}
    \hline
    \textbf{Идентификатор}          & \textbf{Требование}                                                                                                                                                                \\
    \hline
    \endfirsthead
    \hline
    \textbf{Идентификатор}          & \textbf{Требование}                                                                                                                                                                \\
    \hline
    \endhead

    F\_AssetsPage\_1                & Страница активов содержит список всех цифровых активов.                                                                                                                            \\ \hline
    F\_AssetsPage\_2                & Каждый элемент списка (актив) содержит базовую информацию об активе (название актива, название компании, стоимость актива, изменение стоимости актива за последний день).                                                        \\ \hline
    F\_AssetsPage\_3                & На каждом элементе списка (активе) есть кнопка В избранное, позволяющая добавить/удалить актив в избранное.                                                                        \\ \hline
    F\_AssetsPage\_4                & При нажатии на актив открывается страница конкретного актива.                                                                                                                      \\ \hline
    F\_AssetsPage\_5                & За один раз на страницу загружается 10 элементов списка, которые затем могут подгружаться по мере его прокручивания (динамическая загрузка).                                       \\ \hline

    \caption{Функциональные требования для страницы активов}
\end{longtable}

\begin{longtable}{| p{0.35\textwidth} | p{0.65\textwidth} |}
    \hline
    \textbf{Идентификатор}          & \textbf{Требование}                                                                                                                                                                \\
    \hline
    \endfirsthead
    \hline
    \textbf{Идентификатор}          & \textbf{Требование}                                                                                                                                                                \\
    \hline
    \endhead

    F\_AssetPage\_1                 & Страница актива содержит в себе кнопки Назад, В избранное, а также полную информацию об активе (базовая информация + уровень рискованности + описание компании + описание актива + оператор ЦФА).                         \\ \hline
    F\_AssetPage\_2                 & При нажатии на кнопку Назад будет открыт предыдущий экран.                                                                                                                         \\ \hline
    F\_AssetPage\_3                 & При нажатии кнопки В избранное происходит добавление/удаление актива в избранное.                                                                                                  \\ \hline

    \caption{Функциональные требования для страницы конкретного актива}
\end{longtable}

\begin{longtable}{| p{0.35\textwidth} | p{0.65\textwidth} |}
    \hline
    \textbf{Идентификатор}          & \textbf{Требование}                                                                                                                                                                \\
    \hline
    \endfirsthead
    \hline
    \textbf{Идентификатор}          & \textbf{Требование}                                                                                                                                                                \\
    \hline
    \endhead

    F\_FavouritesPage\_1            & Страница избранных активов полностью повторяет страницу активов, за исключением того, что там хранятся только избранные активы пользователя.                                       \\ \hline

    \caption{Функциональные требования для страницы избранных активов}
\end{longtable}

\begin{longtable}{| p{0.35\textwidth} | p{0.65\textwidth} |}
    \hline
    \textbf{Идентификатор}          & \textbf{Требование}                                                                                                                                                                \\
    \hline
    \endfirsthead
    \hline
    \textbf{Идентификатор}          & \textbf{Требование}                                                                                                                                                                \\
    \hline
    \endhead

    F\_PortfoliosPage\_1      & Страница портфелей содержит в себе кнопку Создать портфель, а также список портфелей пользователя.                                                                 \\ \hline
    F\_PortfoliosPage\_2      & Каждый элемент списка (портфель), содержит дату и время создания, уровень рискованности, текущую стоимость, а также изменение стоимости с момента создания портфеля.                                                               \\ \hline
    F\_PortfoliosPage\_3      & При нажатии на кнопку Создать портфель происходит переход на экран создания портфеля.                                                                                  \\ \hline
    F\_PortfoliosPage\_4      & При нажатии на элемент списка (портфель) происходит переход на страницу конкретного портфеля.                                                                           \\ \hline
    F\_PortfoliosPage\_5                & За один раз на страницу загружается 10 элементов списка, которые затем могут подгружаться по мере его прокручивания (динамическая загрузка).                             \\ \hline

    \caption{Функциональные требования для страницы портфелей}
\end{longtable}

\begin{longtable}{| p{0.35\textwidth} | p{0.65\textwidth} |}
    \hline
    \textbf{Идентификатор}          & \textbf{Требование}                                                                                                                                                                \\
    \hline
    \endfirsthead
    \hline
    \textbf{Идентификатор}          & \textbf{Требование}                                                                                                                                                                \\
    \hline
    \endhead

    F\_PortfolioCreatePage\_1 & Страница создания портфеля содержит в себе кнопки Назад, Создать портфель, поле Общая стоимость, радио-кнопку Уровень рискованности.                                                          \\ \hline
    F\_PortfolioCreatePage\_2 & При нажатии на кнопку Назад будет открыт предыдущий экран.                                                                                                                         \\ \hline
    F\_PortfolioCreatePage\_3 & Кнопка Создать портфель неактивна, пока поле Стоимость пустое.                                                                                                                   \\ \hline
    F\_PortfolioCreatePage\_4 & Стоимость должна быть целым числом, не меньшим, чем минимальная стоимость актива выбранной рискованности и не большим, чем 10\_000\_000.                                                                                                                   \\ \hline
    F\_PortfolioCreatePage\_5 & Выбор рискованности является радио-кнопкой, состоящей из выборов Высокорискованный, Среднерискованный, Низкорискованный, Комбинированный.                                            \\ \hline
    F\_PortfolioCreatePage\_6 & При нажатии на кнопку Создать портфель и возникновении ошибки, на экране отображается соответствующее уведомление об ошибке.                           \\ \hline
    F\_PortfolioCreatePage\_7 & При нажатии на кнопку Создать портфель и корректном создании, происходит переход на страницу портфелей.                                                                 \\ \hline
    F\_PortfolioCreatePage\_8 & Портфель создается на основе данных, введенных пользователем, а также данных об активах с помощью специального алгоритма.                                                                 \\ \hline

    \caption{Функциональные требования для страницы создания портфеля}
\end{longtable}

\begin{longtable}{| p{0.35\textwidth} | p{0.65\textwidth} |}
    \hline
    \textbf{Идентификатор}          & \textbf{Требование}                                                                                                                                                                \\
    \hline
    \endfirsthead
    \hline
    \textbf{Идентификатор}          & \textbf{Требование}                                                                                                                                                                \\
    \hline
    \endhead

    F\_PortfolioPage\_1       & Страница портфеля содержит в себе кнопку Назад, а также полную информацию о портфеле (Дата и время создания, Уровень рискованности, Общая стоимость, Изменение стоимости с момента создания портфеля, Список всех цифровых активов портфеля с указанием общей стоимости, изменения стоимости с момента создания портфеля и количества каждого актива).      \\ \hline
    F\_PortfolioPage\_2       & При нажатии на кнопку Назад будет открыт предыдущий экран.                                                                                                                         \\ \hline
    F\_PortfolioPage\_3       & При нажатии на какой-либо актив будет совершен переход на страницу конкретного актива.                                                                                             \\ \hline

    \caption{Функциональные требования для страницы конкретного портфеля}
\end{longtable}

\begin{longtable}{| p{0.35\textwidth} | p{0.65\textwidth} |}
    \hline
    \textbf{Идентификатор}          & \textbf{Требование}                                                                                                                                                                \\
    \hline
    \endfirsthead
    \hline
    \textbf{Идентификатор}          & \textbf{Требование}                                                                                                                                                                \\
    \hline
    \endhead

    F\_SettingsPage\_1              & Экран настроек содержит в себе кнопки Изменить пароль, кнопка с диалоговым окном Тема, поля Cтарый пароль, Новый пароль, Подтверждение пароля, Выйти из аккаунта.                  \\ \hline
    F\_SettingsPage\_2              & Кнопка изменить пароль неактивна, если хотя бы одно из полей Старый пароль, Новый пароль, Подтверждение пароля пустое, а также если пользователь вошел через Google.        \\ \hline
    F\_SettingsPage\_3              & При нажатии на кнопку Изменить пароль и ошибки изменения пароля, на экране отображается соответствующее уведомление об ошибке.                                                     \\ \hline
    F\_SettingsPage\_4              & При нажатии на кнопку Изменить пароль и успешном изменении пароля, на экране отображается соответствующее уведомление.                                                             \\ \hline
    F\_SettingsPage\_5              & При нажатии на кнопку Переключить тему появляется диалоговое окно с предложением выбора темы (Системная, Темная, Светла).                                                                                               \\ \hline
    F\_SettingsPage\_6              & При нажатии на кнопку выйти из аккаунта происходит выход из аккаунта и переход на страницу входа.                                                                                   \\ \hline
    F\_SettingsPage\_7              & Старый пароль должен совпадать с текущим паролем пользователя.                                                                                   \\ \hline
    F\_SettingsPage\_8              & Поля Новый пароль и Подтверждение пароля должны совпадать.                                                                                   \\ \hline

    \caption{Функциональные требования для страницы настроек}
\end{longtable}

\begin{longtable}{| p{0.35\textwidth} | p{0.65\textwidth} |}
    \hline
    \textbf{Идентификатор}          & \textbf{Требование}                                                                                                                                                                \\
    \hline
    \endfirsthead
    \hline
    \textbf{Идентификатор}          & \textbf{Требование}                                                                                                                                                                \\
    \hline
    \endhead

    F\_Navigation\_1                & Между страницами активов, избранных активов, портфелей, настроек навигация происходит с помощью нижнего меню.                                                                 \\ \hline
    F\_RiskDegree\_1    & Каждый актив имеет свой уровень рискованности, рассчитываемый на основе информации об эмитенте и об активе с помощью специального алгоритма.                                                  \\ \hline

    \caption{Остальные функциональные требования}
\end{longtable}

\subsubsection{Требования к надежности}

\begin{itemize}
    \item Сервер, обрабатывающий бизнес-логику приложения должен обладать высокой доступностью уровня \texttt{99\% (two nines)}, то есть быть недоступным для обработки запросов не более $3.65$ суток в год.
    \item Клиентское приложение должно корректно обрабатывать возможные ошибки, не завершая аварийно свою работу.
\end{itemize}

\subsubsection{Условия эксплуатации}

Специальные условия эксплуатации не требуются.
Обслуживание не требуется.

\subsubsection{Требования к составу и параметрам технических средств}

Требования к серверу:
\begin{itemize}
    \item Поддержка виртуализации,
    \item Доступ в интернет,
    \item Программное обеспечение: Docker.
\end{itemize}

\noindent Требования к устройству клиента:
\begin{itemize}
    \item Мобильное устройство на операционной системе Android,
    \item Версия Android SDK не менее 26 (Android 8.0),
    \item Доступ в интернет,
    \item Актуальная версия сервисов Google (опционально, используются для входа через аккаунт Google).
\end{itemize}

\subsubsection{Требования к информационной и программной совместимости}

Серверное приложение:
\begin{itemize}
    \item Язык программирования -- Java,
    \item Веб-фреймворк -- Spring,
    \item СУБД -- PostgreSQL.
\end{itemize}

\noindent Клиентское приложение:
\begin{itemize}
    \item Язык программирования -- Kotlin,
    \item Версия Android SDK не менее 26 (Android 8.0),
    \item Актуальная версия сервисов Google (опционально, используются для входа через аккаунт Google).
\end{itemize}

\subsubsection{Требования к маркировке и упаковке}

Клиентское приложение собирается в APK-архив для дальнейшего
распространения и установки на мобильные устройства.
Маркировка не требуется.

\subsubsection{Требования к транспортированию и хранению}

Специальные требования к транспортировке и хранению не предъявляются.

\subsubsection{Специальные требования}

\begin{itemize}
    \item Клиентское приложение должно поддерживать следующие языки: русский, английский;
    \item Пользовательский интерфейс приложения должен быть разработан в соответствии с дизайн-системой Material 3.
\end{itemize}

\subsection{Требования к программной документации}

\noindent Предварительный состав программной документакции:
\begin{itemize}
    \item функциональная спецификация,
    \item high-level design,
    \item спецификация Application Programming Interface,
    \item описание используемых алгоритмов,
    \item план модульного тестирования,
    \item план интеграционного тестирования,
    \item план end-to-end тестирования,
    \item код программы.
\end{itemize}

\noindent Функциональная спецификация должна содержать функциональные и нефункциональные требования к разрабатываемому приложению.
В документации High-Level Design должны быть представлены: макет дизайна интерфейса приложения, архитектура приложения,
стек используемых технологий для frontend и backend частей приложения, диаграмма классов, схема базы данных и API. Все
схемы должны сопровождаться описанием.


\subsection{Технико-экономические показатели}
Ориентировочная экономическая эффективность приложения: Учитывая, что данное приложение разрабатывается в рамках
учебного проекта, ориентировочная экономическая эффективность не только заключается в финансовых показателях, но и в
педагогических целях. Процесс разработки приложения предоставляет студентам возможность практически применить полученные
знания и навыки в области разработки Android приложений. В качестве дополнительной экономической выгоды можно
рассматривать возможность привлечения инвестиций на будущие проекты или участие в конкурсах и выставках студенческих проектов.

Предпологаемая годовая потребность: Годовая потребность в использовании учебного приложения будет зависеть от его
актуальности и востребованности среди студентов и преподавателей. Предполагается, что в первый год эксплуатации
приложения оно будет использоваться студентами из различных курсов и направлений, что может привести к формированию
постоянной аудитории в последующие годы.

Экономические преимущества разработки по сравнению с лучшими отечественными и зарубежными образцами или аналогами:
Одним из главных экономических преимуществ учебного проекта является возможность минимизировать затраты на разработку
приложения благодаря использованию свободных или студенческих лицензий для программного обеспечения и инструментов
разработки. Кроме того, такой проект предоставляет возможность для студентов проявить свои навыки и творческий
потенциал, что в долгосрочной перспективе может привести к их успешной карьере в сфере информационных технологий.

\subsection{Стадии и этапы разработки}

\noindent Проект имел следующие стадии разработки:
\begin{itemize}
    \item Разработка функциональной спецификации - Требуется сформулировать функциональные и нефункциональные требования на разрабатываемый программный продукт.,
    \item Разработка High-Level Desigh(HLD) - Требуется разработать High-Level Design (HLD). В документе должны быть описаны макет дизайна интерфейса (при наличии), общая архитектура приложения, стэк технологий, спроектированы диаграммы классов/модулей/etc, схемы баз данных, API. Все схемы должны сопровождаться описанием.,
    \item Разработка программного продукта - Необходимо разработать программный продукт, опираясь на FS и HLD.,
    \item Доработка программного продукта - Необходимо доработать программный продукт, в том числе: 
    \begin{itemize}\item Произвести рефакторинг кода. \item Протестировать работу программного продукта в строгом соответствии со всеми пунктами FS. \item Добавить новые фичи.\end{itemize}
    \item Покрыть код модульными тестами - Необходимо покрыть код приложения модульными тестами. Покрытие кода должно составлять не менее 80 процентов.,
    \item Покрыть код интеграционными тестами,
    \item Покрыть код системными тестами,
\end{itemize}

\subsection{Порядок контроля и приемки}

\subsubsection{Виды испытаний}

Проект проверялся следующими видами испытаний:
\begin{itemize}
    \item Необходимо выполнить модульное тестирование ПО, покрытие должно составлять более 80%
    \item Необходимо выполнить интеграционное тестирование нескольких модулей. Минимальное количество сценариев - 10, сценарии должны быть согласованы с руководителем.
    \item Необходимо протестировать ПО по бизнес-требованиям, сформированным перед началом проектирования ПО. В качестве тестовых сценариев выбираются основные сценарии использования ПО в полнолстью рабочем окружении.
\end{itemize}

\subsubsection{Общие требования к приемке работы}

\begin{itemize}
    \item Обязательно наличие репозитория на любом хостинге(github/gitlab/bitbucket/etc). Репозиторий ведется согласно github flow/gitlab flow/git flow(По выбору команды).
    \item Система контроля за разработкой, где есть возможность создать kanban-доску - любая из Github Projects, Trello, Notion 
\end{itemize}

\end{document}