\documentclass[a4paper, 14pt]{article}
\usepackage[margin=1.6cm]{geometry}
\usepackage[utf8]{inputenc}
\usepackage{minted}
\usepackage[russian]{babel}
\usepackage{amsmath}
\usepackage{graphicx}
\usepackage{changepage}
\usepackage{hyperref}
\usepackage{cases}
\usepackage{tikz-timing}[2017/12/20]
\usepackage{relsize}
\usepackage{booktabs}
\usepackage{gensymb}
\usepackage{multirow}
\usepackage{longtable}
\usetikzlibrary {arrows.meta}

\hypersetup{
	linkbordercolor = {1 1 1}
}

\usepackage{tikz-timing}[2009/05/15]
\usepackage{multicol}
\usepackage[T2A]{fontenc}
\usepackage{pgfplots}
%\usepackage[left=2.5cm, right=1.5cm, vmargin=2.5cm]{geometry}
\setlength\parindent{0pt} % Удалить отступы из параграфов.

\usepackage{listings}
\usepackage{caption}
\DeclareCaptionFont{white}{\color{white}} % Текст заголовка.
\DeclareCaptionFormat{listing}{\colorbox{gray}{\parbox{\textwidth}{#1#2#3}}}
\captionsetup[lstlisting]{format=listing,labelfont=white,textfont=white}
\renewcommand\labelenumi{\theenumi)}
\setlength\parindent{24pt}



\begin{document}
\lstset{
    language=java,                 % Выбор языка для подсветки (здесь это java).
    basicstyle=\small\sffamily,    % Размер и начертание шрифта для подсветки кода.
    numbers=left,                  % Где поставить нумерацию строк (слева\справа).
    numberstyle=\tiny,             % Размер шрифта для номеров строк.
    stepnumber=1,                  % Размер шага между двумя номерами строк.
    firstnumber=1,
    numberfirstline=true
    numbersep=5pt,                 % Как далеко отстоят номера строк от подсвечиваемого кода.
    backgroundcolor=\color{white}, % Цвет фона подсветки - используем \usepackage{color}.
    showspaces=false,              % Показывать или нет пробелы специальными отступами.
    showstringspaces=false,        % Показывать или нет пробелы в строках.
    showtabs=false,                % Показывать или нет табуляцию в строках.
    frame=single,                  % Рисовать рамку вокруг кода.
    tabsize=2,                     % Размер табуляции по умолчанию равен 2 пробелам.
    captionpos=t,                  % Позиция заголовка вверху [t] или внизу [b].
    breaklines=true,               % Автоматически переносить строки (да\нет).
    breakatwhitespace=false,       % Переносить строки только если есть пробел.
    escapeinside={\%*}{*)}         % Если нужно добавить комментарии в коде.
}

\begin{titlepage}
    \center

    ФЕДЕРАЛЬНОЕ ГОСУДАРСТВЕННОЕ АВТОНОМНОЕ ОБРАЗОВАТЕЛЬНОЕ УЧРЕЖДЕНИЕ ВЫСШЕГО ОБРАЗОВАНИЯ\linebreak
    «Санкт-Петербургский политехнический университет Петра Великого»
    \noindent\rule{500pt}{0.8pt} \\
    \textsc{\Large Институт компьютерных наук и кибербезопасности}\\
    \textsc{\large Высшая школа программной инженерии}\\[1.5cm]

    { \huge \bfseries ФУНКЦИОНАЛЬНАЯ СПЕЦИФИКАЦИЯ	\\
    \Large \mdseries АГРЕГАТОР ЦИФРОВЫХ ФИНАНСОВЫХ АКТИВОВ <<ТЕССЕРАКТ>> \\
    \large по дисциплине <<Технологии разработки качественного программного обеспечения>>}\\
    \flushright{
        {\phantom{qwe}}\\[1.0cm]
    }

    \begin{figure}[H]
        \centering
        \includegraphics[width=10cm]{1.png}\\[2.0cm]
    \end{figure}

    \begin{multicols}{2}
        \begin{flushright} \large

            {Выполнили студенты группы: 5130904/00104:}\\
            {\phantom{qwe}}\\
            {\phantom{qwe}}\\
            {\phantom{qwe}}\\
            {\phantom{qwe}}\\

            {Преподаватель:\\}

        \end{flushright}
        \begin{flushright}

            {Почернин В. С.}\\
            {Шиляев В. С.}\\
            {Мурзаканов И. М.}\\
            {Разукрантов В. Е.}\\[0.5cm]


            Маслаков А. П.\\

        \end{flushright}
    \end{multicols}

    \flushright{
        {\phantom{qwe}}\\[0.5cm]
    }
    \centering{
        Санкт-Петербург\\
        2023
    }

    \vfill
\end{titlepage}

\Large
\tableofcontents
\newpage
\large

\section{Функциональная спецификация}

\subsection{Список страниц}

\begin{enumerate}
    \item Страница входа (LoginPage).
    \item Страница информации (InfoPage).
    \item Страница регистрации (RegistrationPage).
    \item Страница активов (AssetsPage).
    \item Страница конкретного актива (AssetPage).
    \item Страница избранных активов (FavouritesPage).
    \item Страница портфелей (PortfoliosPage).
    \item Страница создания портфеля (PortfolioCreatePage).
    \item Страница конкретного портфеля (PortfolioPage).
    \item Страница настроек (SettingsPage).
\end{enumerate}

\subsection{Функциональные требования}

\subsubsection{Страница входа (LoginPage)}

\begin{longtable}{| p{0.35\textwidth} | p{0.65\textwidth} |}
    \hline
    \textbf{Идентификатор}          & \textbf{Требование}                                                                                                                                                                \\
    \hline
    \endfirsthead
    \hline
    \textbf{Идентификатор}          & \textbf{Требование}                                                                                                                                                                \\
    \hline
    \endhead

    F\_LoginPage\_1                 & Страница входа содержит логотип приложения, кнопки Войти, Войти с помощью Google, Зарегистрироваться, Информация, поля Логин, Пароль.                                                  \\ \hline
    F\_LoginPage\_2                 & При нажатии на кнопку Информация происходит переход на станицу информации.                                                                                                         \\ \hline
    F\_LoginPage\_3                 & Кнопка Войти неактивна, пока хотя бы одно из полей Логин или Пароль пустое.                                                                                                        \\ \hline
    F\_LoginPage\_4                 & При нажатии на кнопку Войти и возникновении ошибки аутентификации, на экране отображается соответствущее уведомление об ошибке.                                                    \\ \hline
    F\_LoginPage\_5                 & При нажатии на кнопку Войти и корректной аутентификации происходит переход на страницу активов.                                                                                    \\ \hline
    F\_LoginPage\_6                 & При нажатии на кнопку Войти через Google и возникновении ошибки аутентификации, на экране отображается соответствущее уведомление об ошибке.                                       \\ \hline
    F\_LoginPage\_7                 & При нажатии на кнопку Войти через Google и корректной аутентификации через Google OAuth API происходит переход на страницу активов.                                                \\ \hline
    F\_LoginPage\_8                 & При нажатии на кнопку Регистрация происходит переход на страницу регистрации.                                                                                                      \\ \hline

    \caption{Функциональные требования для страницы входа}
\end{longtable}

\subsubsection{Страница информации (InfoPage)}

\begin{longtable}{| p{0.35\textwidth} | p{0.65\textwidth} |}
    \hline
    \textbf{Идентификатор}          & \textbf{Требование}                                                                                                                                                                \\
    \hline
    \endfirsthead
    \hline
    \textbf{Идентификатор}          & \textbf{Требование}                                                                                                                                                                \\
    \hline
    \endhead

    F\_InfoPage\_1                  & Страница информации содержит информацию о приложении, кнопку Назад.                                                                                                                \\ \hline
    F\_InfoPage\_2                  & При нажатии на кнопку Назад будет открыт предыдущий экран.                                                                                                                         \\ \hline

    \caption{Функциональные требования для страницы информации}
\end{longtable}

\subsubsection{Страница регистрации (RegistrationPage)}

\begin{longtable}{| p{0.35\textwidth} | p{0.65\textwidth} |}
    \hline
    \textbf{Идентификатор}          & \textbf{Требование}                                                                                                                                                                \\
    \hline
    \endfirsthead
    \hline
    \textbf{Идентификатор}          & \textbf{Требование}                                                                                                                                                                \\
    \hline
    \endhead

    F\_RegistrationPage\_1          & Страница регистрации содержит кнопки Назад, Зарегистрироваться, поля Логин, Email, Пароль, Подтверждение пароля.                                                                   \\ \hline
    F\_RegistrationPage\_2          & При нажатии на кнопку Назад будет открыт предыдущий экран.                                                                                                                         \\ \hline
    F\_RegistrationPage\_3          & Кнопка Зарегистрироваться неактивна, пока хотя бы одно из полей Логин, Email, Пароль, Подтверждение пароля пустое.                                                                 \\ \hline
    F\_RegistrationPage\_4          & Длина пароля должна составлять от 6ти до 30ти символов включительно.                                                                                                               \\ \hline
    F\_RegistrationPage\_5          & Пароль может состоять из цифр, латинских букв верхнего и нижнего регистра, а также специальных символов.                                                                           \\ \hline
    F\_RegistrationPage\_6          & Специальными символами являются: <<! @ \# \$ \% \& * ( ) - \_ + = ; : , . / ? $\backslash$ | [ ] \{ \}>>.                                                                          \\ \hline
    F\_RegistrationPage\_7          & Пароль должен содержать хотя бы одну букву (в любом регистре), а также хотя бы одну цифру или специальны символ (т.е. Буква И (Цифра ИЛИ Специальный символ)).                     \\ \hline
    F\_RegistrationPage\_8          & При нажатии на кнопку зарегистрироваться и возникновении ошибки регистрации, на экране отображается соответствующее уведомление об ошибке.                                         \\ \hline
    F\_RegistrationPage\_9          & При нажатии на кнопку зарегистрироваться и успешной регистрации происходит переход на страницу входа, а на экране отображается уведомление об успешной регистрации.                \\ \hline
    F\_RegistrationPage\_10         & Длина логина должна составлять от 3х до 16ти символов включительно.                                                                                                                \\ \hline
    F\_RegistrationPage\_11         & Логин может состоять из цифр и латинских букв верхнего и нижнего регистра.                                                                                                         \\ \hline

    \caption{Функциональные требования для страницы регистрации}
\end{longtable}

\subsubsection{Страница активов (AssetsPage)}

\begin{longtable}{| p{0.35\textwidth} | p{0.65\textwidth} |}
    \hline
    \textbf{Идентификатор}          & \textbf{Требование}                                                                                                                                                                \\
    \hline
    \endfirsthead
    \hline
    \textbf{Идентификатор}          & \textbf{Требование}                                                                                                                                                                \\
    \hline
    \endhead

    F\_AssetsPage\_1                & Страница активов содержит список всех цифровых активов.                                                                                                                            \\ \hline
    F\_AssetsPage\_2                & Каждый элемент списка (актив) содержит базовую информацию об активе (название актива, название компании, стоимость актива, изменение стоимости актива за последний месяц).                                                        \\ \hline
    F\_AssetsPage\_3                & На каждом элементе списка (активе) есть кнопка В избранное, позволяющая добавить/удалить актив в избранное.                                                                        \\ \hline
    F\_AssetsPage\_4                & При нажатии на актив открывается страница конкретного актива.                                                                                                                      \\ \hline
    F\_AssetsPage\_5                & За один раз на страницу загружается 10 элементов списка, которые затем могут подгружаться по мере его прокручивания (динамическая загрузка).                                       \\ \hline

    \caption{Функциональные требования для страницы активов}
\end{longtable}

\subsubsection{Страница конкретного актива (AssetPage)}

\begin{longtable}{| p{0.35\textwidth} | p{0.65\textwidth} |}
    \hline
    \textbf{Идентификатор}          & \textbf{Требование}                                                                                                                                                                \\
    \hline
    \endfirsthead
    \hline
    \textbf{Идентификатор}          & \textbf{Требование}                                                                                                                                                                \\
    \hline
    \endhead

    F\_AssetPage\_1                 & Страница актива содержит в себе кнопки Назад, В избранное, а также полную информацию об активе (базовая информация + уровень рискованности + описание компании + описание актива).                         \\ \hline
    F\_AssetPage\_2                 & При нажатии на кнопку Назад будет открыт предыдущий экран.                                                                                                                         \\ \hline
    F\_AssetPage\_3                 & При нажатии кнопки В избранное происходит добавление/удаление актива в избранное.                                                                                                  \\ \hline

    \caption{Функциональные требования для страницы конкретного актива}
\end{longtable}

\subsubsection{Страница избранных активов (FavouritesPage)}

\begin{longtable}{| p{0.35\textwidth} | p{0.65\textwidth} |}
    \hline
    \textbf{Идентификатор}          & \textbf{Требование}                                                                                                                                                                \\
    \hline
    \endfirsthead
    \hline
    \textbf{Идентификатор}          & \textbf{Требование}                                                                                                                                                                \\
    \hline
    \endhead

    F\_FavouritesPage\_1            & Страница избранных активов полностью повторяет страницу активов, за исключением того, что там хранятся только избранные активы пользователя.                                       \\ \hline

    \caption{Функциональные требования для страницы избранных активов}
\end{longtable}

\subsubsection{Страница портфелей (PortfoliosPage)}

\begin{longtable}{| p{0.35\textwidth} | p{0.65\textwidth} |}
    \hline
    \textbf{Идентификатор}          & \textbf{Требование}                                                                                                                                                                \\
    \hline
    \endfirsthead
    \hline
    \textbf{Идентификатор}          & \textbf{Требование}                                                                                                                                                                \\
    \hline
    \endhead

    F\_PortfoliosPage\_1      & Страница портфелей содержит в себе кнопку Создать портфель, а также список портфелей пользователя.                                                                 \\ \hline
    F\_PortfoliosPage\_2      & Каждый элемент списка (портфель), содержит дату и время создания, рискованность, а также стоимость портфеля.                                                               \\ \hline
    F\_PortfoliosPage\_3      & При нажатии на кнопку Создать портфель происходит переход на экран создания портфеля.                                                                                  \\ \hline
    F\_PortfoliosPage\_4      & При нажатии на элемент списка (портфель) происходит переход на страницу конкретного портфеля.                                                                           \\ \hline
    F\_PortfoliosPage\_5                & За один раз на страницу загружается 10 элементов списка, которые затем могут подгружаться по мере его прокручивания (динамическая загрузка).                             \\ \hline

    \caption{Функциональные требования для страницы портфелей}
\end{longtable}

\subsubsection{Страница создания портфеля (PortfolioCreatePage)}

\begin{longtable}{| p{0.35\textwidth} | p{0.65\textwidth} |}
    \hline
    \textbf{Идентификатор}          & \textbf{Требование}                                                                                                                                                                \\
    \hline
    \endfirsthead
    \hline
    \textbf{Идентификатор}          & \textbf{Требование}                                                                                                                                                                \\
    \hline
    \endhead

    F\_PortfolioCreatePage\_1 & Страница создания портфеля содержит в себе кнопки Назад, Создать портфель, поле Стоимость, радио-кнопку Cтепень рискованности.                                                          \\ \hline
    F\_PortfolioCreatePage\_2 & При нажатии на кнопку Назад будет открыт предыдущий экран.                                                                                                                         \\ \hline
    F\_PortfolioCreatePage\_3 & Кнопка Создать портфель неактивна, пока поле Стоимость пустое.                                                                                                                   \\ \hline
    F\_PortfolioCreatePage\_4 & Стоимость должна быть целым числом, не меньшим, чем минимальная стоимость актива выбранной рискованности и не большим, чем 1\_000\_000.                                                                                                                   \\ \hline
    F\_PortfolioCreatePage\_5 & Выбор рискованности является радио-кнопкой, состоящей из выборов Высокорискованный, Среднерискованный, Консервативный, Комбинированный.                                            \\ \hline
    F\_PortfolioCreatePage\_6 & При нажатии на кнопку Создать портфель и возникновении ошибки, на экране отображается соответствующее уведомление об ошибке.                           \\ \hline
    F\_PortfolioCreatePage\_7 & При нажатии на кнопку Создать портфель и корректном создании, происходит переход на страницу портфелей.                                                                 \\ \hline
    F\_PortfolioCreatePage\_8 & Портфель создается на основе данных, введенных пользователем, а также данных об активах с помощью специального алгоритма.                                                                 \\ \hline

    \caption{Функциональные требования для страницы создания портфеля}
\end{longtable}

\subsubsection{Страница конкретного портфеля (PortfolioPage)}

\begin{longtable}{| p{0.35\textwidth} | p{0.65\textwidth} |}
    \hline
    \textbf{Идентификатор}          & \textbf{Требование}                                                                                                                                                                \\
    \hline
    \endfirsthead
    \hline
    \textbf{Идентификатор}          & \textbf{Требование}                                                                                                                                                                \\
    \hline
    \endhead

    F\_PortfolioPage\_1       & Страница портфеля содержит в себе кнопку Назад, а также полную информацию о портфеле (Дата и время создания, Общая стоимость, Уровень рискованности, Список всех цифровых активов портфеля с указанием общей стоимости и количества каждого актива).      \\ \hline
    F\_PortfolioPage\_2       & При нажатии на кнопку Назад будет открыт предыдущий экран.                                                                                                                         \\ \hline
    F\_PortfolioPage\_3       & При нажатии на какой-либо актив будет совершен переход на страницу конкретного актива.                                                                                             \\ \hline

    \caption{Функциональные требования для страницы конкретного портфеля}
\end{longtable}

\subsubsection{Страница настроек (SettingsPage)}

\begin{longtable}{| p{0.35\textwidth} | p{0.65\textwidth} |}
    \hline
    \textbf{Идентификатор}          & \textbf{Требование}                                                                                                                                                                \\
    \hline
    \endfirsthead
    \hline
    \textbf{Идентификатор}          & \textbf{Требование}                                                                                                                                                                \\
    \hline
    \endhead

    F\_SettingsPage\_1              & Экран настроек содержит в себе кнопки Изменить пароль, кнопка с диалоговым окном Тема, поля Cтарый пароль, Новый пароль, Подтверждение нового пароля, Выйти из аккаунта.                  \\ \hline
    F\_SettingsPage\_2              & Кнопка изменить пароль неактивна, если хотя бы одно из полей Старый пароль, Новый пароль, Подтверждение нового пароля пустое, а также если пользователь вошел через Google.        \\ \hline
    F\_SettingsPage\_3              & При нажатии на кнопку Изменить пароль и ошибки изменения пароля, на экране отображается соответствующее уведомление об ошибке.                                                     \\ \hline
    F\_SettingsPage\_4              & При нажатии на кнопку Изменить пароль и успешном изменении пароля, на экране отображается соответствующее уведомление.                                                             \\ \hline
    F\_SettingsPage\_5              & При нажатии на кнопку Переключить тему появляется диалоговое окно с предложением выбора темы (Системная, Темная, Светла).                                                                                               \\ \hline
    F\_SettingsPage\_6              & При нажатии на кнопку выйти из аккаунта происходит выход из аккаунта и переход на страницу входа.                                                                                   \\ \hline

    \caption{Функциональные требования для страницы настроек}
\end{longtable}

\subsubsection{Остальные функциональные требования}

\begin{longtable}{| p{0.35\textwidth} | p{0.65\textwidth} |}
    \hline
    \textbf{Идентификатор}          & \textbf{Требование}                                                                                                                                                                \\
    \hline
    \endfirsthead
    \hline
    \textbf{Идентификатор}          & \textbf{Требование}                                                                                                                                                                \\
    \hline
    \endhead

    F\_Navigation\_1                & Между страницами активов, избранных активов, портфелей, настроек навигация происходит с помощью нижнего меню.                                                                 \\ \hline
    F\_RiskDegree\_1    & Каждый актив имеет свой уровень рискованности, рассчитываемую на основе информации об эмитенте и об активе с помощью специального алгоритма.                                                  \\ \hline

    \caption{Остальные функциональные требования}
\end{longtable}

\subsection{Нефункциональные требования}

\begin{longtable}{| p{0.3\textwidth} | p{0.7\textwidth} |}
    \hline
    \textbf{Идентификатор} & \textbf{Требование}                                                                                               \\
    \hline
    \endfirsthead
    \hline
    \textbf{Идентификатор} & \textbf{Требование}                                                                                               \\
    \hline
    \endhead

    % You can also use the below code for more things
    % \hline
    % \endfoot
    % \hline
    % \endlastfoot

    NF\_PasswordStore      & Пароли должны храниться в базе данных в зашифрованном виде.                                                       \\ \hline
    NF\_ClientServer       & Приложение должно иметь клиент-серверную архитектуру.                                                             \\ \hline
    NF\_OS                 & Приложение должно работать на мобильной ОС Android.                                                               \\ \hline
    NF\_AssetsStore        & Хранение информации об акивах должно иметь унифицированный вид для возможности удобного добавления новых активов. \\ \hline
    NF\_Design             & Внешний вид мобильного приложения должен быь выполнен в соответствии с Material Design 3.                         \\ \hline
    NF\_Language           & Приложение должно работать либо на русском, либо на английском языке, в зависимости от настроек системы.          \\ \hline
    NF\_ServerLanguage     & Серверная часть приложения должна быть написана на языке Java.                                                    \\ \hline
    NF\_Database     & В качестве СУБД должна быть использована PostgreSQL.                                                    \\ \hline
    NF\_Currency     & В качестве валюты в приложении используется Российский рубль (RUB).                                                  \\ \hline
    \caption{Нефункциональные требования}
\end{longtable}

\end{document}